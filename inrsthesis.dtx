% \iffalse meta-comment
%
% Copyright (C) 2017 by Joey Dumont <joey.dumont@gmail.com>
% -------------------------------------------------------
%
% This file may be distributed and/or modified under the
% conditions of the LaTeX Project Public License, either version 1.3
% of this license or (at your option) any later version.
% The latest version of this license is in:
%
%    http://www.latex-project.org/lppl.txt
%
% and version 1.3 or later is part of all distributions of LaTeX
% version 2005/12/01 or later.
%
% \fi
%
% \iffalse
%<*driver>
\ProvidesFile{inrsthesis.dtx}
%</driver>
%<class>\NeedsTeXFormat{LaTeX2e}[1999/12/01]
%<class>\ProvidesClass{inrsthesis}
%<*class>
    [2017/06/22 v1.0 INRS-EMT thesis template]
%</class>
%
%<*driver>
\documentclass[11pt]{ltxdoc}
  \usepackage{ifxetex}
  \ifxetex
    \RequirePackage{amsmath}
    \RequirePackage{fontspec}
    \RequirePackage{unicode-math}
    \defaultfontfeatures{Ligatures=TeX}
  \else
    \RequirePackage[T1]{fontenc}
    \RequirePackage[utf8]{inputenc}
  \fi
  \usepackage{natbib}
  \usepackage{color,booktabs,metalogo}
  \usepackage{todonotes}

  \EnableCrossrefs
  \CodelineIndex
  \RecordChanges

  \definecolor{link}{rgb}{0,0.4,0.6}   % ~RoyalBlue de dvips
  \definecolor{url}{rgb}{0.6,0,0}      % rouge-brun
  \definecolor{citation}{rgb}{0,0.5,0} % vert foncé

  \usepackage{hyperref}
  \hypersetup{colorlinks, linktocpage,
    urlcolor=url, linkcolor=link, citecolor=citation,
    bookmarksopen, bookmarksnumbered, bookmarksdepth=subsection.
    pdftitle={INRS Class Template Reference Manual},
    pdfauthor={Joey Dumont}}

\begin{document}
  \DocInput{inrsthesis.dtx}
\end{document}
%</driver>
% \fi
%
% \CheckSum{0}
%
% \CharacterTable
%  {Upper-case    \A\B\C\D\E\F\G\H\I\J\K\L\M\N\O\P\Q\R\S\T\U\V\W\X\Y\Z
%   Lower-case    \a\b\c\d\e\f\g\h\i\j\k\l\m\n\o\p\q\r\s\t\u\v\w\x\y\z
%   Digits        \0\1\2\3\4\5\6\7\8\9
%   Exclamation   \!     Double quote  \"     Hash (number) \#
%   Dollar        \$     Percent       \%     Ampersand     \&
%   Acute accent  \'     Left paren    \(     Right paren   \)
%   Asterisk      \*     Plus          \+     Comma         \,
%   Minus         \-     Point         \.     Solidus       \/
%   Colon         \:     Semicolon     \;     Less than     \<
%   Equals        \=     Greater than  \>     Question mark \?
%   Commercial at \@     Left bracket  \[     Backslash     \\
%   Right bracket \]     Circumflex    \^     Underscore    \_
%   Grave accent  \`     Left brace    \{     Vertical bar  \|
%   Right brace   \}     Tilde         \~}
%
% \changes{v1.0}{2017/06/22}{Initial version}
%
% \GetFileInfo{inrsthesis.dtx}
%
% \DoNotIndex{\newcommand,\newenvironment}
%
%
% \title{\textsf{inrsthesis}: A Template for INRS-ÉMT Theses. \thanks{This document
%   corresponds to \textsf{inrsthesis}~\fileversion, dated \filedate.}}
% \author{Joey Dumont \\ \texttt{joey.dumont@gmail.com}}
%
% \maketitle
%
% \section{Introduction}
%
% This class aims to create a {\LaTeX} document which obeys the INRS thesis format
% guidelines set forth \href{http://sdis.inrs.ca/sites/sdis.inrs.ca/files/INRSGuideST-Rev2016-fr.pdf}{here}.
% Most of the functionality of this class is based on the |ulthese| class.
%
% \section{Usage}
%
% \StopEventually
% ^^A INDEX DOES NOT WORK \StopEventually{\PrintChanges\PrintIndex}
%
% \newpage
% \appendix
%
% \section{Implementation}
% \label{sec:implementation}
%
% This appendix goes through the implementation of the class.
%
% Conditionals are very useful in general. In this case, we use
% \textsc{ifxetex} to support the Xe{\LaTeX} engine.
%    \begin{macrocode}
%<*class>
\RequirePackage{ifthen}
\RequirePackage{ifxetex}
\RequirePackage{xpatch}
%    \end{macrocode}
%
% \subsection{Class Options}

% \begin{macro}{10pt,11pt,12pt}
%
% This defines the default font size of the document. When this option is processed,
% |\normalsize| is defined to be either |10pt|, |11pt| or |12pt|. Font sizes
% primitives such as |\LARGE| and |\small| are adjusted accordingly.
%
% This option is passed to \textsf{memoir} to ensure that page elements
% are always the proper size. It is also stored in |\INRS@ptsize| for
% later use (see |\maketitle|).
%    \begin{macrocode}
\newcommand*{\INRS@ptsize}{}
\DeclareOption{10pt}{%
  \PassOptionsToClass{10pt}{memoir}
  \renewcommand*{\INRS@ptsize}{10}}
\DeclareOption{11pt}{%
  \PassOptionsToClass{11pt}{memoir}
  \renewcommand*{\INRS@ptsize}{11}}
\DeclareOption{12pt}{%
  \PassOptionsToClass{12pt}{memoir}
  \renewcommand*{\INRS@ptsize}{12}}
%    \end{macrocode}
% \end{macro}
%
% \begin{macro}{SymmetricalJury}
%
% This changes the format of the |\jury| information. While
% the option |SymmetricalJury| does not exactly follow the INRS guidelines,
% it is more pleasing to the eye.
%    \begin{macrocode}
\newboolean{INRS@SymmetricalJury}
\setboolean{INRS@SymmetricalJury}{false}
\DeclareOption{SymmetricalJury}{%
  \setboolean{INRS@SymmetricalJury}{true}%
  \newlength{\INRS@jurycolsep}%
  \setlength{\INRS@jurycolsep}{20pt}%
  }
%    \end{macrocode}
% \end{macro}
%
% \subsection{Loading the \textsf{memoir} class}
%
% All remaining class options are passed to \textsc{memoir}.
% We explicitly pass |twoside| and |openright|, as those are required
% by the INRS thesis format guidelines.
%    \begin{macrocode}
\DeclareOption*{\PassOptionsToClass{\CurrentOption}{memoir}}
\ExecuteOptions{11pt,letterpaper}
\ProcessOptions
\LoadClass[twoside,openright]{memoir}
%    \end{macrocode}
%
% \subsection{Required Pacakages}
%
% Here we describe the packages that are needed to use |inrsthesis|.
%
%
% \textsc{graphicx} is a ubiquitously used {\LaTeX} package to include
% graphics in a document. We load it here, along with \textsc{xcolor}
% for user convenience.
%    \begin{macrocode}
\RequirePackage{graphicx}
\RequirePackage{xcolor}
\RequirePackage{tikz}
%    \end{macrocode}
%
% The |\textcopyright| command used on the title page is provided by the
% \textsf{textcomp} package.
%    \begin{macrocode}
\RequirePackage{textcomp}
%    \end{macrocode}
%
% \subsection{Hyperlink colour}
% \label{sec:couleurs}
%
% La classe définit une couleur standard pour les hyperliens, une
% teinte de bleu assez foncée pour être à fois visible en couleur et
% peu contrastante si le document est imprimé en noir et blanc.
%    \begin{macrocode}
\definecolor{INRSlinkcolor}{rgb}{0,0,0.3}
%    \end{macrocode}
%
% \subsection{Margins}
%
% The INRS guidelines demand uniform margins of 25~mm, on both odd and
% even pagess. While the same guidelines state the folio should be located
% 10pt from the bottom of the page, this is way too close and does not even
% replicate the Word template. Here, we use a distance of 10mm, as in
% the |ulthese| class.
%    \begin{macrocode}
\setlrmarginsandblock{25mm}{25mm}{*}
\setulmarginsandblock{25mm}{25mm}{*}
\checkandfixthelayout[nearest]
\setlength{\footskip}{\lowermargin}
\addtolength{\footskip}{-10mm}
%    \end{macrocode}
%
% Since the front matter of thesis can contain multiple pages,
% it can happen that they overfill the boxes in the ToC. We thus
% make the boxes bigger.
%    \begin{macrocode}
\renewcommand{\@pnumwidth}{3em}
\renewcommand{\@tocrmarg}{4em}
%    \end{macrocode}
%
% \subsection{Line Spacing}
%
% The line spacing is fixed to one half spacing. We do not use spaces
% between paragraphs, but rather a first-line indent.
%    \begin{macrocode}
\OnehalfSpacing
\setlength{\parskip}{0em}
\setlength{\parindent}{2em}
%    \end{macrocode}
%
% However, the ToC, list of tables and list of figures are singly spaced.
%    \begin{macrocode}
\renewcommand{\tocheadstart}{\SingleSpacing\chapterheadstart}
\renewcommand{\lotheadstart}{\SingleSpacing\chapterheadstart}
\renewcommand{\lofheadstart}{\SingleSpacing\chapterheadstart}
%    \end{macrocode}
%
% \subsection{Header and footer}
%
% The header and footer are empty across the whole document,
% except for the folio in the centre of the footer. We alias
% the |plain| pagestyle to our own |inrs| pagestyle such that
% the front and back matters use it by default.
%    \begin{macrocode}
\makepagestyle{inrs}
  \makeevenfoot{inrs}{}{\thepage}{}
  \makeoddfoot{inrs}{}{\thepage}{}
\aliaspagestyle{plain}{inrs}
\pagestyle{inrs}
%    \end{macrocode}
%
% \subsection{Title Page}
%
% Most of the package is devoted to composing the title page, which is achieved
% through a redefinition of |\maketitle|.
%
% \subsubsection{Font family}
%
% For simple vanity reasons, we use Helvetica on the title page. Here,
% |\fontencoding{T1}| ensures that both the {\XeLaTeX} and {\LaTeX} engines
% load the same font.
%
%    \begin{macrocode}
\newcommand*{\INRS@phvfamily}{\fontencoding{T1}\fontfamily{phv}\selectfont}
%    \end{macrocode}
% Here, we use our previously defined |\INRS@ptsize| to make sure that the
% elements of the title page are always the same size. For a reference,
% check Table~3.9 of the \href{http://muug.ca/mirror/ctan/macros/latex/contrib/memoir/memman.pdf}{\textsf{memoir} manual}.
%    \begin{macrocode}
\ifnum\INRS@ptsize=10\relax
  \newcommand*{\INRS@fonttitle}{\normalfont\LARGE\bfseries\INRS@phvfamily}
  \newcommand*{\INRS@fontsubtitle}{\normalfont\Large\bfseries\INRS@phvfamily}
  \newcommand*{\INRS@fontauthor}{\normalfont\Large\INRS@phvfamily}
  \newcommand*{\INRS@fontprogram}{\INRS@fontauthor}
  \newcommand*{\INRS@fontbase}{\normalfont\Large\INRS@phvfamily}
\fi
\ifnum\INRS@ptsize=11\relax
  \newcommand*{\INRS@fonttitle}{\normalfont\Large\bfseries\INRS@phvfamily}
  \newcommand*{\INRS@fontsubtitle}{\normalfont\large\bfseries\INRS@phvfamily}
  \newcommand*{\INRS@fontauthor}{\normalfont\large\INRS@phvfamily}
  \newcommand*{\INRS@fontprogram}{\INRS@fontauthor}
  \newcommand*{\INRS@fontbase}{\normalfont\large\INRS@phvfamily}
\fi
\ifnum\INRS@ptsize=12\relax
  \newcommand*{\INRS@fonttitle}{\normalfont\large\bfseries\INRS@phvfamily}
  \newcommand*{\INRS@fontsubtitle}{\normalfont\normalsize\bfseries\INRS@phvfamily}
  \newcommand*{\INRS@fontauthor}{\normalfont\normalsize\bfseries\INRS@phvfamily}
  \newcommand*{\INRS@fontprogram}{\INRS@fontauthor}
  \newcommand*{\INRS@fontbase}{\normalfont\normalsize\INRS@phvfamily}
\fi
%    \end{macrocode}
% \subsubsection{Title page API}
%
% We use a set of internal and user facing commands to build the title page.
% The internal commands are defined below.
%    \begin{macrocode}
\newcommand{\INRS@maintitle}{}
\newcommand{\INRS@subtitle}{}
\newcommand*{\INRS@author}{}
\newcommand*{\INRS@program}{}
\newcommand*{\INRS@year}{}
\newcommand*{\INRS@centre}{}
\newcommand*{\INRS@jury}{}
%    \end{macrocode}

%\begin{macro}{INRS@printjury}
% The jury is a little complex to typeset if one wants to have
% both columns left-aligned and still be symmetrical enough to be
% pleasing to the eye . The |SymmetricalJury| class option right aligns the
% first column. The user facing |\juryitem| is used as a building block
% of |\jury|, defined below.
%
% |\juryitem| takes two arguments. The first is the content of the first column,
% while the second is the content of the second column.
%
% The code for |SymmetricalJury| is courtesy of user egreg on StackOverflow.
% See the original \href{https://tex.stackexchange.com/questions/376501/centering-a-tabular-environment-such-that-whitespace-is-equally-distributed}{question} for details.
%    \begin{macrocode}
\newcommand{\juryitem}[2]{}
\newlength{\INRS@jurywidth}

\newcommand{\INRS@printjury}{%
  \begin{center}
  % this just prints the divider line (remove in the final version)
  %\makebox[\textwidth]{%
  %  \smash{\vrule width 0.1pt depth \textheight height 0pt}%
  %}
  %%% end part to remove
  \sbox0{%
    \def\juryitem##1##2{\begin{tabular}{@{}l@{}}##1\end{tabular}}%
    \begin{tabular}{@{}l@{}}\INRS@jury\end{tabular}%
  }%
  \setlength{\INRS@jurywidth}{\wd0}%
  \sbox0{%
    \def\juryitem##1##2{\begin{tabular}{@{}l@{}}##2\end{tabular}}%
    \begin{tabular}{@{}l@{}}\INRS@jury\end{tabular}%
  }%
  \ifdim\wd0>\INRS@jurywidth \setlength{\INRS@jurywidth}{\wd0}\fi
  \renewcommand{\arraystretch}{3}%
  \def\juryitem##1##2{%
    \renewcommand\arraystretch{1}%
    \begin{tabular}[t]{@{}l@{}}##1\end{tabular}
    &
    \renewcommand\arraystretch{1}%
    \begin{tabular}[t]{@{}l@{}}##2\end{tabular}
  }%
  % Actual printing of the jury.
  \ifthenelse{\boolean{INRS@SymmetricalJury}}
  {
    {
      \setlength{\tabcolsep}{\INRS@jurycolsep}
      \begin{tabular}{@{}>{\raggedleft}p{7cm}>{\raggedright\arraybackslash}p{7cm}@{}}
      \INRS@jury
      \end{tabular}
    }
  }
  {
    \begin{tabular}{@{}p{\INRS@jurywidth}@{\hspace{1cm}}p{\INRS@jurywidth}@{}}
    \INRS@jury
    \end{tabular}
  }
  \end{center}
}
%    \end{macrocode}
% \end{macro}
% User-facing API to modify the internal title page commands.
% Note that since the subtitle is optional, we store its presence
% or its absence in a boolean value to better typeset the title page.
% The |\title| is always typeset is capital letters. This is enforced with
% with the |\uppercase| {\TeX} macro, as the {\LaTeX} equivalent |\MakeUpperCase|
% had trouble with line breaks.
% The |\program| macro is automatically converted to lower case.
% The |\jury| should be formatted as a series of |\juryitem|s.
%
%    \begin{macrocode}
\newboolean{INRS@hassubtitle}
\renewcommand{\title}[1]{\renewcommand{\INRS@maintitle}{#1}}
\newcommand{\subtitle}[1]{%
  \setboolean{INRS@hassubtitle}{true}
  \renewcommand{\INRS@subtitle}{#1}}
\renewcommand*{\author}[1]{\renewcommand*{\INRS@author}{#1}}
\renewcommand*{\year}[1]{\renewcommand*{\INRS@year}{#1}}
\newcommand*{\program}[1]{\renewcommand*{\INRS@program}{#1}}
\newcommand{\centreINRS}[1]{\renewcommand*{\INRS@centre}{#1}}
\newcommand{\jury}[1]{\renewcommand*{\INRS@jury}{#1}}
%    \end{macrocode}
%
% \subsubsection{Title and subtitle}
%
% To obtain a consistent positioning of the elements of the title page,
% we place the title and subtitle in boxes. This enables us to compute
% the height of the title+subtitle block and adjust the distance
% between the title and author blocks.
%
% \begin{macro}{\INRS@measuretitle}
%   The title and subtitles are placed in the boxes
%   |\INRS@titlebox| et |\INRS@subtitlebox|. The
%   |\INRS@measuretitle| macro allows the measurement of the typeset block later on.
%   We set a line spacing of 1.5 in this block.
%    \begin{macrocode}
\newsavebox{\INRS@titlebox}
\newsavebox{\INRS@subtitlebox}
\newlength{\INRS@titleboxtotht}
\newlength{\INRS@subtitleboxtotht}
\newcommand{\INRS@measuretitle}{%
  \setbox\INRS@titlebox=\vbox{%
    \centering\INRS@fonttitle\expandafter\uppercase\expandafter{\INRS@maintitle}}
  \setlength{\INRS@titleboxtotht}{%
    \dimexpr\ht\INRS@titlebox+\dp\INRS@titlebox}
  \ifthenelse{\boolean{INRS@hassubtitle}}{%
    \setbox\INRS@subtitlebox=\vbox{%
      \centering\vspace*{\baselineskip}\INRS@fontsubtitle\INRS@subtitle}
    \setlength{\INRS@subtitleboxtotht}{%
      \dimexpr\ht\INRS@subtitlebox+\dp\INRS@subtitlebox}}{}}
%    \end{macrocode}
% \end{macro}
%
% \subsubsection{Building the Title Page}
%
% \begin{macro}{\pagetitre}
%   We use a single line spacing and no space between paragraphs.
%
%   The first block consists in the name of the university,
%   which is hard-coded in the |\maketitle| command. We then
%   add the specific Centre from which the thesis originates through
%   the |\centreINRS| macro.
%
%   The second block is title+subtitle block. The title is centered
%   and capitalized. The subtitle is in bold and centered.
%
%   The third block is the author block.
%
%   The fourth block is the program block. We state whether the thesis
%   is submitted for a Masters or doctoral degree. We then state the program.
%
%   The next block is the jury, which lists all of the referees of the thesis.
%   We provide two options to typeset the jury list. The default one is the
%   one that most resembles the Word template, i.e. both columns are left-aligned.
%   The class makes sure that this configuration is as symmetrical as possible, but
%   having both columns left-aligned makes this hard. The second option makes
%   the first column right-aligned, which makes for ann equal amount of whitespace
%   between the columns and makes for a  more aesthestically pleasing typeset.
%
%   The last block is simply the copyright, the author and the year.
%
%    \begin{macrocode}
\newlength{\INRS@docidspacing}
\setlength{\INRS@docidspacing}{45pt}
\newlength{\INRS@authorspacing}
\setlength{\INRS@authorspacing}{45pt}
\renewcommand{\maketitle}{{%
    \clearpage
    \thispagestyle{empty}
    \SingleSpacing\setlength{\parskip}{0pt}
    {\centering\INRS@fontbase
      Universit\'e du Qu\'ebec\\
      Institut National de la Recherche Scientifique\\
      \INRS@centre\\}
    \vspace{\INRS@docidspacing}
    \centering
    \INRS@measuretitle
    \addtolength{\INRS@docidspacing}{-\INRS@titleboxtotht}
    \addtolength{\INRS@docidspacing}{-\INRS@subtitleboxtotht}
    \ifdim\INRS@docidspacing<\baselineskip\relax
        \setlength{\INRS@docidspacing}{\baselineskip}
        \addtolength{\INRS@authorspacing}{-\baselineskip}
    \fi
    \box\INRS@titlebox
    \box\INRS@subtitlebox
    \vspace{\INRS@authorspacing}
    {\centering\INRS@fontbase
     Par \\
     \INRS@author}\\
    \vspace{\INRS@authorspacing}
    {\centering\INRS@fontprogram
     Thèse présentée pour l'obtention du grade de\\
     Philosophi\ae~ Doctor (Ph.~D.)\\
     en \expandafter\MakeLowercase\expandafter{\INRS@program}}

     \vfill

     {\INRS@fontsubtitle Jury d'évaluation}
     \vspace{-10pt}
    {\INRS@fontbase\INRS@printjury}

     \vspace*{10pt}

     \begin{flushleft}
    {\textcopyright} {\INRS@fontbase Droits réservés de \INRS@author, \INRS@year\par}
    \cleardoublepage
    \end{flushleft}
    }}
%    \end{macrocode}
% \end{macro}
%
% \subsection{Chapter styles}
%
% In this section we define the chapter styles of the front and main matters.
% The front matter has unnumbered and centered titles with point size 16. The
% main chapters are typeset in all caps, and are left-aligned, but have a point
% size of only 14.
%
% \subsubsection{Front Matter}
%
%    \begin{macrocode}
\makechapterstyle{INRSFrontMatter}
{
  \renewcommand{\chapterheadstart}{}
  \renewcommand{\printchaptername}{}
  \renewcommand{\chapternamenum}{}
  \renewcommand{\printchapternum}{}
  \renewcommand{\afterchapternum}{}
  \renewcommand{\printchaptertitle}[1]{\centering\chaptitlefont \MakeUppercase{##1}\par}
  \renewcommand{\chaptitlefont}{\normalfont\LARGE\bfseries\INRS@phvfamily}
  \setlength{\beforechapskip}{0pt}
}
\xapptocmd{\frontmatter}{%
  \chapterstyle{INRSFrontMatter}
}
{}{}
%    \end{macrocode}
%
%    \begin{macrocode}
\makechapterstyle{INRSMainMatter}
{
  \renewcommand{\chapterheadstart}{}
  \renewcommand{\printchaptername}{}
  \renewcommand{\chapternamenum}{}
  \renewcommand{\printchapternum}{\chapnumfont \thechapter}
  \renewcommand{\afterchapternum}{\hspace*{\midchapskip}}
  \renewcommand{\printchaptertitle}[1]{\chaptitlefont \MakeUppercase{##1}}
  \renewcommand{\chapnumfont}{\normalfont\Large\bfseries}
  \renewcommand{\chaptitlefont}{\normalfont\Large\bfseries}
  \setlength{\beforechapskip}{0pt}
  \setlength{\midchapskip}{2em}
  \setlength{\afterchapskip}{40pt}
}
\xapptocmd{\mainmatter}{%
  \chapterstyle{INRSMainMatter}
  \pagestyle{inrs}
}
{}{}
%    \end{macrocode}
% \subsection{Appendices}
%
% \begin{macro}{\appendix}
%   If the thesis contains any appendices, they should be numbered with capital
%   Roman numerals.
%    \begin{macrocode}
\xapptocmd{\appendix}{\gdef\thechapter{\@Roman\c@chapter}}{}{}
%    \end{macrocode}
% \end{macro}
%
% \subsection{Dedication and epigraph}
%
% Those macros are straight up lifted from |ulthese|. We use
% the |epigraph| command from \textsc{memoir} to typeset both
% |\dedication| and |epigraph|.
% \begin{macro}{\epigraph}
%   To safely define a speciliazed |\epigraph| command, we use the |letltxmacro|
%   package.
%    \begin{macrocode}
\xpretocmd{\epigraph}{%
    \clearpage
    \pagestyle{empty}
    \setlength{\beforeepigraphskip}{10\baselineskip}
    \mbox{}
}
{}{}
%    \end{macrocode}
% \end{macro}
% \begin{macro}{\dedication}
%   The dedication is typeset on its own page, and is simply an epigraph
%   without an author.
%    \begin{macrocode}
\newcommand{\dedication}[1]{{%
    %\clearpage
    %\pagestyle{empty}
    \setlength{\epigraphrule}{0pt}
    \epigraphtextposition{flushright}
    \epigraph{\itshape #1}{}}}
%    \end{macrocode}
% \end{macro}
%
%</class>
%
% ^^A End of the class implementation.
%
% \Finale
