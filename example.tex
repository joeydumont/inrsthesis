% --------------------------------------------------------------------------- %
% Author:          Joey Dumont                <joey.dumont@gmail.com>         %
% Date created:    Sep. 21st, 2018                                            %
% Description:     Example file for the inrsthesis class.                     %
% License:         CC0                                                        %
%                  <https://creativecommons.org/publicdomain/zero/1.0>        %
% --------------------------------------------------------------------------- %

% --------------------------------------------------------------------------- %
% --                               Preamble                                -- %
% --------------------------------------------------------------------------- %

% ----------------------------------------------------------------- %
% --                       Document Class                        -- %
% ----------------------------------------------------------------- %

% -- Define the documentclass (see class documentation for options).
\documentclass[11pt,SymmetricalJury,PhD]{inrsthesis}  % For Ph. D. theses
%\documentclass[11pt,SymmetricalJury,MSc]{inrsthesis} % For M. Sc. theses
\usepackage{lipsum}

% ----------------------------------------------------------------- %
% --                          Packages                           -- %
% ----------------------------------------------------------------- %

% These packages are entirely optional. I use there here to customize
% the output to my liking: you can use whatever package you want here.

% -- Math symbols and fonts.
\let\mathfrak\undefined
\usepackage[charter]{mathdesign}      % Bitstream Charter for best font.
\usepackage[no-math]{fontspec}
\setmainfont{EB Garamond}
\usepackage[varg]{newtxmath}
% \usepackage{unicode-math}

\usepackage{microtype}   % Fine small typographical details
\usepackage{realscripts} % Use the font's sub- and superscripts

\usepackage[unicode=true,
      pdfauthor={Joey Dumont},
      pdftitle={inrsthesis -- an example file},
      bookmarks=true,
      bookmarksnumbered=true,
      bookmarksopen=true,
      bookmarksopenlevel=1,
      bookmarksdepth=1,
      breaklinks=false,
      pdfborder={0 0 0},
      backref=false,
      colorlinks=true,
      linktoc=page,
      linkcolor=red,
      citecolor=blue,
      urlcolor=blue]
           {hyperref}

\usepackage[numbers,square,sort&compress]{natbib}
\usepackage{subcaption}
\usepackage{todonotes}

% ----------------------------------------------------------------- %
% --                        Customization                        -- %
% ----------------------------------------------------------------- %

% Again, these are option.
\newenvironment{chaptersummary}{%
  \begin{quotation}
  \SingleSpacing
  \setlength{\parskip}{\baselineskip}}{%
  \end{quotation}}

\allowdisplaybreaks

\settocdepth{section}
\setsecnumdepth{subsection}

%\captionsetup[figure]{labelfont=bf,width=0.85\textwidth}
%\captionsetup[table]{labelfont=bf,width=0.85\textwidth}

% Caption fix with new TexLive version.
\makeatletter
\renewcommand{\counterwithin}{\@ifstar{\@csinstar}{\@csin}}
\makeatother

% Configuration of subcaptions.
\renewcommand\thesubfigure{\alph{subfigure}}
\renewcommand\thesubtable{\alph{subtable}}
  \captionsetup[figure]{width=0.90\textwidth,labelfont={sf,bf},textfont=sf,font+=small}
  \captionsetup[subfigure]{width=0.8\linewidth,parskip=0pt,font+=small}
  \captionsetup[table]{width=0.90\textwidth,labelfont={sf,bf},textfont=sf,font+=small}
  \captionsetup[subtable]{width=\linewidth,parskip=0pt,font+=small}

\usetikzlibrary{tikzmark,calc}

% ----------------------------------------------------------------- %
% --                       Title Page Info                       -- %
% ----------------------------------------------------------------- %

% This section MUST be in the preamble.

\title{inrsthesis -- an example file}
%\subtitle{Towards a Realistic Modelling of High-Power \\Laser Systems in the
% Quantum Theory}
\author{Joey Dumont}
\year{2018}
\program{Sciences de l'énergie et des matériaux}
\centreINRS{Centre Énergie Matériaux et Télécommunications}
\jury{
  \juryitem
    {Président du jury et \\ examinateur interne}
    {Nom du professeur \\ Institution}
  \\
  \juryitem
    {Examinateur externe}
    {Nom du professeur \\ Faculté ou département \\ Institution}
  \\
  \juryitem
    {Examinateur interne}
    {Nom du professeur \\ Institution}
  \\
  \juryitem
    {Directeur de recherche}
    {Nom du professeur \\ Institution}
  \\
  \juryitem
    {Codirecteur de recherche}
    {Nom du professeur \\ Institution \\ Institution}
}

% --------------------------------------------------------------------------- %
% --                               Document                                -- %
% --------------------------------------------------------------------------- %

\begin{document}

\frontmatter

\maketitle

\chapter{Résumé}

\textbf{Mots-clés}:

\chapter{Abstract}

\textbf{Keywords}:

\chapter{Sommaire récapitulatif}

This part is only necessary if you write your thesis in English.

\cleardoublepage

\tableofcontents
\cleardoublepage

\listoftables
\cleardoublepage

\listoffigures
\cleardoublepage

\dedication{I dedicate this to Mufasa, the God of Lions.}
\cleardoublepage

\epigraph{Education, \textit{n}: that which discloses to the wise and disguises
from the foolish their lack of understanding.}{Ambrose Bierce}

\cleardoublepage

\mainmatter

\chapter{Introduction}

\lipsum[1]

\section{First section of introduction}

\lipsum[2]

\begin{figure}
  \centering
  \missingfigure[figcolor=white]{Something or other.}
  \caption[Caption goes here.]{This is a nice and cogent description
           of the data presented in the figure. The results sure are
           convincing!}
  \label{fig:test}
\end{figure}

\lipsum[3]

  \begin{equation}
    \label{eq:nice-equation}
    \chi = \frac{|e|\hbar\sqrt{\left(F^{\mu\nu}p_\nu\right)^2}}{c^3m_e^3} = \frac{(pk)}{(m_e c\omega)}\frac{E_0}{E_S}
  \end{equation}

Ain't Eq.~\eqref{eq:nice-equation} nice? See Appendix \ref{app.technical} for
painstaking detail!

\chapter{Another chapter}
\label{chapter:methods}

\section{First section}

\subsection{First subsection}

\subsection{Second subsection}


\begin{table}
  \centering
  \begin{tabular}{lrr}
  \toprule
                       & \multicolumn{1}{c}{MP2}   & \multicolumn{1}{c}{GP3}      \\
  \midrule
  Base nodes           & AMD Opteron 6172          & Intel E5-2683 v4 (Broadwell) \\
  Number of base nodes & 1632                      & 864                          \\
  Processors per nodes & 24                        & 32                           \\
  RAM per node         & 32 GiB                    & 128 GiB                      \\
  Interconnect         & InfiniBand QDR (10 Gb/s)  & InfiniBand EDR (100 Gb/s)    \\
  \bottomrule
  \end{tabular}
  \caption{Summary of the properties of the MP2 and GP3 Compute Canada supercomputing clusters.}
  \label{tab:summary-of-cc-clusters}
\end{table}


\appendix
\addtocontents{toc}{\protect\setcounter{tocdepth}{0}}
%\addtocontents{toc}{\addtolength\cftchapternumwidth{\widthof{Appendix\;}}}
\renewcommand{\thechapter}{\Alph{chapter}}
\renewcommand{\thesection}{\Alph{chapter}.\arabic{section}}

\chapter{Technical Result in Painstaking Detail}
\label{app.technical}

\backmatter

% -- Bibliography in a nice style.
\nocite{*}
\bibliographystyle{osajnl}
\bibliography{thesis}

\end{document}
